% Options for packages loaded elsewhere
\PassOptionsToPackage{unicode}{hyperref}
\PassOptionsToPackage{hyphens}{url}
%
\documentclass[
]{article}
\usepackage{lmodern}
\usepackage{amssymb,amsmath}
\usepackage{ifxetex,ifluatex}
\ifnum 0\ifxetex 1\fi\ifluatex 1\fi=0 % if pdftex
  \usepackage[T1]{fontenc}
  \usepackage[utf8]{inputenc}
  \usepackage{textcomp} % provide euro and other symbols
\else % if luatex or xetex
  \usepackage{unicode-math}
  \defaultfontfeatures{Scale=MatchLowercase}
  \defaultfontfeatures[\rmfamily]{Ligatures=TeX,Scale=1}
\fi
% Use upquote if available, for straight quotes in verbatim environments
\IfFileExists{upquote.sty}{\usepackage{upquote}}{}
\IfFileExists{microtype.sty}{% use microtype if available
  \usepackage[]{microtype}
  \UseMicrotypeSet[protrusion]{basicmath} % disable protrusion for tt fonts
}{}
\makeatletter
\@ifundefined{KOMAClassName}{% if non-KOMA class
  \IfFileExists{parskip.sty}{%
    \usepackage{parskip}
  }{% else
    \setlength{\parindent}{0pt}
    \setlength{\parskip}{6pt plus 2pt minus 1pt}}
}{% if KOMA class
  \KOMAoptions{parskip=half}}
\makeatother
\usepackage{xcolor}
\IfFileExists{xurl.sty}{\usepackage{xurl}}{} % add URL line breaks if available
\IfFileExists{bookmark.sty}{\usepackage{bookmark}}{\usepackage{hyperref}}
\hypersetup{
  pdftitle={Reproducible Research: Peer Assessment 1},
  hidelinks,
  pdfcreator={LaTeX via pandoc}}
\urlstyle{same} % disable monospaced font for URLs
\usepackage[margin=1in]{geometry}
\usepackage{graphicx,grffile}
\makeatletter
\def\maxwidth{\ifdim\Gin@nat@width>\linewidth\linewidth\else\Gin@nat@width\fi}
\def\maxheight{\ifdim\Gin@nat@height>\textheight\textheight\else\Gin@nat@height\fi}
\makeatother
% Scale images if necessary, so that they will not overflow the page
% margins by default, and it is still possible to overwrite the defaults
% using explicit options in \includegraphics[width, height, ...]{}
\setkeys{Gin}{width=\maxwidth,height=\maxheight,keepaspectratio}
% Set default figure placement to htbp
\makeatletter
\def\fps@figure{htbp}
\makeatother
\setlength{\emergencystretch}{3em} % prevent overfull lines
\providecommand{\tightlist}{%
  \setlength{\itemsep}{0pt}\setlength{\parskip}{0pt}}
\setcounter{secnumdepth}{-\maxdimen} % remove section numbering

\title{Reproducible Research: Peer Assessment 1}
\author{}
\date{\vspace{-2.5em}}

\begin{document}
\maketitle

\hypertarget{loading-and-preprocessing-the-data}{%
\subsection{Loading and preprocessing the
data}\label{loading-and-preprocessing-the-data}}

\begin{verbatim}
activity <- read.csv("activity/activity.csv", colClasses = c("numeric", "character", 
    "numeric"))
head(activity)
\end{verbatim}

Load lattice library for plotting

\begin{verbatim}
library(lattice)
\end{verbatim}

Cast dates to year-month-day

\begin{verbatim}
activity$date <- as.Date(activity$date,"%Y-%m-d%")
\end{verbatim}

\hypertarget{what-is-mean-total-number-of-steps-taken-per-day}{%
\subsection{What is mean total number of steps taken per
day?}\label{what-is-mean-total-number-of-steps-taken-per-day}}

First get the total steps per day using the aggregate function

\begin{verbatim}
total <- aggregate(steps ~ date,activity,sum,na.rm=TRUE)
\end{verbatim}

Plot the histogram showing the totals per day

\begin{verbatim}
hist(total$steps, main = "Total steps by day", xlab = "day")
\end{verbatim}

The mean is calculated as follows:

\begin{verbatim}
mean(total$steps)
\end{verbatim}

The median is calculated as follows:

\begin{verbatim}
median(total$steps)
\end{verbatim}

\hypertarget{what-is-the-average-daily-activity-pattern}{%
\subsection{What is the average daily activity
pattern?}\label{what-is-the-average-daily-activity-pattern}}

To get the time series, first we will use the following to group the
data:

\begin{verbatim}
series <- tapply(activity$steps, activity$interval, mean, na.rm = TRUE)
\end{verbatim}

Then we plot the plot as shown below:

\begin{verbatim}

plot(row.names(series), series, type = "l", xlab = "5-minute interval", ylab = "average across all the days", main = "Time series plot of the average number of steps taken")
\end{verbatim}

The 5-minute interval that, on average, contains the maximum number of
steps:

\begin{verbatim}
max <- which.max(series)
names(max)
\end{verbatim}

\hypertarget{imputing-missing-values}{%
\subsection{Imputing missing values}\label{imputing-missing-values}}

This section describes the code used to impute the missing values

First, fetch all the missing values:

\begin{verbatim}
actNA <- sum(is.na(activity))
\end{verbatim}

The number of missing values are:

\begin{verbatim}
actNA
\end{verbatim}

To impute the missing values, we will use the mean:

\begin{verbatim}
average <- aggregate(steps ~ interval, activity,mean)
fillNA <- numeric()
for (i in 1:nrow(activity)) {
    obs <- activity[i, ]
    if (is.na(obs$steps)) {
        steps <- subset(average, interval == obs$interval)$steps
    } else {
        steps <- obs$steps
    }
    fillNA <- c(fillNA, steps)
}
\end{verbatim}

Make a copy of the activity field and fill the missing with the mean:

\begin{verbatim}
activity_1 <- activity
activity_1$steps <- fillNA
\end{verbatim}

Histogram of the total number of steps taken each day after missing
values are imputed:

\begin{verbatim}
total_1 <- aggregate(steps ~ date, activity_1, sum, na.rm = TRUE)
hist(total_1$steps, main = "Total steps by day", xlab = "day")
\end{verbatim}

\hypertarget{are-there-differences-in-activity-patterns-between-weekdays-and-weekends}{%
\subsection{Are there differences in activity patterns between weekdays
and
weekends?}\label{are-there-differences-in-activity-patterns-between-weekdays-and-weekends}}

We use the weekdays function here:

\begin{verbatim}
weekday <- weekdays(activity$date)
day_type <- vector()
for (i in 1:nrow(activity)) {
    if (weekday[i] == "Saturday") {
        day_type[i] <- "Weekend"
    } else if (weekday[i] == "Sunday") {
        day_type[i] <- "Weekend"
    } else {
        day_type[i] <- "Weekday"
    }
}
\end{verbatim}

Set the day type on the data frame:

\begin{verbatim}
activity$day_type <- day_type
activity$day_type <- factor(activity$day_type)

per_day_steps <- aggregate(steps ~ interval + day_type, activity, mean)
names(per_day_steps) <- c("interval", "day_type", "steps")
\end{verbatim}

Panel plot comparing the average number of steps taken per 5-minute
interval across weekdays and weekends:

\begin{verbatim}
xyplot(steps ~ interval | day_type, per_day_steps, type = "l", layout = c(2, 1),        xlab = "Interval", ylab = "Number of steps")
\end{verbatim}

\end{document}
